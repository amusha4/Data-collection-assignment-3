\documentclass[14pt]{article}

\usepackage{graphicx}

\begin{document}
\title{A REPORT ABOUT THE CONSUMPTION OF DIFFERENT FOODS BY THE YOUTH}
\author{A. SHARON}

\date {\today}

\maketitle

\tableofcontents

\section{INTRODUCTION}

Different foods are consumed by different people. When it comes to campus students, they consume at different rates and at different prices. Consumption of some foods is quite costly while others are manageable. This makes the work of the consumers and the suppliers easy.

\section{BACKGROUND}

This topic came up in order to enable some suppliers to know the preferences of the campus students so that their market is enhanced and they are able to benefit from it and as well impress their market. 

\section{OBJECTIVES}

The objectives of this project are that both parties benefit from it that is the suppliers and the consumers. The consumers should be able to get good quality products and not the defective ones which may lead to sicknesses.
Another objective is to see to it that the prices of the different foods consumed by some campus students is affordable and hygienic.

\section{PROBLEM STATEMENT}

The problem lies in the prices of the foods. Some of them are affordable while others are quite expensive for example a plate of chips and chicken at a certain place in town is at 10000 Uganda shillings which is quite expensive and thus some campus students are not able to live by those standards. Another problem with it is some of the foods consumed may be defective and some consumers don’t mind about the expiry dates of some of the foods for example powdered milk and some of the foods can easily get aerated and become tasteless.
Some of the foods require preservation for example yoghurt which needs to be preserved in a cold place and some campus students don’t have the refrigerators where they can preserve their drinks. Some foods are unhealthy and may cause sicknesses and hence putting the life of the consumer at a risk.

\section{FUNCTIONALITY AND SCREEN SHOTS AND RESULTS}

The data was collected electronically using the Open Data Kit Application which involves questionnaires that were used to interview the different consumers of the different types of foods and some of the screenshots show their replies.

\begin{figure}[h!]
\includegraphics[width=100mm,scale=0.5]{1.jpg}
\caption{app engine.}
\label{figure1}
\end{figure}

\begin{figure}[h!]
\includegraphics[width=100mm,scale=0.5]{3.jpg}
\caption{odk form.}
\label{figure2}
\end{figure}


\begin{table}[h]
\centering
\begin{tabular}{c c }
\hline
TYPE OF FOOD &	PRICE \\[0.5ex] 
\hline
FRIES AND CHICKEN & 10000\\
CASSAVA &	100 EACH STICK\\ 
RICE &	3000\\ 
\hline
\end{tabular}
\end{table}

\section{PROCEDURE}

This project involved asking different people questions about what they consumed in a particular day and how it affected them. Some people had the positive things and as well the negative when it came to the prices which make the cost of living high and the way the foods were delivered to them.
The data was collected electronically which involved the location of the area and for this case the Google maps were used and also different pictures of the different foods were taken. It involved connecting my forms to the server by inserting a url of my app engine which resulted to my results being displayed on the screen.

\section{CONCLUSION}

It was quite a tiring experience wen collecting the data because I faced some problems when collecting the data in that some people were not ready to share their information with me. However I was able to get some information from some cooperative people but all in all the project was quite adventuring.

\begin{thebibliography}{10}

\bibitem{latexGuide} The Campusers

Available at \texttt{The community around Makerere University }

\bibitem{latexGuide} The hostels

Available at \texttt{Areas around Wandegeya }W



\end{thebibliography}



\end{document}






